En este Trabajo Fin de Grado (\gls{TFG}) se expone el diseño e implementación de una herramienta web para el análisis, procesamiento y visualización de señales del electroencefalograma (\gls{EEG}). Este trabajo es la continuación del \gls{TFG}: \textit{Análisis de \gls{EEG} para la evaluación de las emociones} \cite{tesis-hugo}, en el que se describen las técnicas y procedimientos para el procesamiento y análisis de señales del \gls{EEG}, y que está implementado completamente en Matlab. \\

El objetivo principal subyacente a la realización de este trabajo es la obtención de una herramienta de \textit{software libre} que permita el cálculo de medidas objetivas a partir del procesamiento del electroencefalograma. En este trabajo, las medidas objetivas son los siguientes cuatro índices: \textit{Índice de atención, Índice de memorización, Índice de aproximación-rechazo e Índice de compromiso}. \\

Además, otro de los objetivos de este trabajo es que la herramienta proporcione la visualización de todas las señales y de todos los valores intermedios que se utilizan en todas las fases del cálculo de estos índices.\\

Esta herramienta web está implementada completamente en el entorno matemático R, tanto la parte del servidor \textit{(backend)} como la parte de interfaz de usuario \textit{(frontend)}. Para ello se ha hecho uso del framework web Shiny. \\

Para la obtención de los índices, el primer problema planteado es el entendimiento e interiorización de todo el \gls{TFG} \cite{tesis-hugo}, desde los conceptos médicos del \gls{EEG} hasta el cálculo final de los índices. El segundo problema consiste en aprender el framework Shiny, sus características y cómo funciona para poder realizar la implementación de la herramienta. \\

El tercero es estudiar y analizar distintas técnicas de filtrado de señales, así como distintas herramientas para la visualización de las mismas. El cuarto problema es realizar la implementación de todo el procedimiento para el cálculo de los índices. Y por último, realizar una evaluación de los resultados obtenidos en comparación con los obtenidos en \cite{tesis-hugo}.


