% Chapter Template

\chapter{Hipótesis de trabajo y objetivos} % Main chapter title
\label{chap:job-hypothesis} % Change X to a consecutive number; for referencing this chapter elsewhere, use \ref{ChapterX}


%------------------------------------------------    
\section{Hipótesis de trabajo}
Los cosméticos son unos de los productos que más se consumen en los hogares, hasta tal punto que en algunos casos se han vuelto imprescindibles. Sin embargo, gran parte de los cosméticos presentan en su composición productos químicos dañinos para las personas. \\

Con esta idea en mente, la hipótesis de este trabajo es aplicar técnicas de Inteligencia Artificial y Big Data al sector de la cosmética para poder encontrar patrones y relaciones entre los cosméticos y los productos químicos que contienen, así como encontrar una clasificación de dichos productos químicos para poder tener un mayor control sobre de qué están compuestos los cosméticos que consumimos. \\




%------------------------------------------------    
\section{Objetivos}
Los objetivos principales de este trabajo son:
\begin{itemize}
 \item Encontrar una clasificación de los productos químicos presentes en los cosméticos.
 \item Encontrar qué productos químicos son los más frecuentes en los cosméticos, así como los cosméticos que mayor productos químicos presentan.
 \item Obtener una predicción de la cantidad de productos químicos dañinos que contendrán los futuros cosméticos.
\end{itemize}
























