% Chapter Template

\chapter{Hipótesis de trabajo y objetivos} % Main chapter title
\label{chap:job-hypothesis} % Change X to a consecutive number; for referencing this chapter elsewhere, use \ref{ChapterX}


%------------------------------------------------    
\section{Introducción}
En este capítulo se va a exponer la hipótesis de este trabajo así como los objetivos propuestos para el mismo.



%------------------------------------------------    
\section{Hipótesis de trabajo}

Los cosméticos son unos de los productos que más se consumen en los hogares, hasta tal punto que en algunos casos se han vuelto imprescindibles. Muchos productos químicos diferentes se utilizan en la fabricación de cosméticos. Los grupos de defensa de consumidores y trabajadores están preocupados porque algunos productos cosméticos contienen sustancias químicas que se sabe o se sospecha que causan cáncer, defectos de nacimiento o daños al sistema reproductivo. Aquellos que trabajan con cosméticos, incluidos los peluqueros, los estilistas y el cuidado de la piel, el cuidado del cuerpo y los trabajadores de salones de uñas, pueden ser más vulnerables a los efectos adversos para la salud que presentan estos productos porque manejan con mayor frecuencia grandes cantidades de cosméticos \citep{dataset}. \\

Con esta idea en mente, la hipótesis de este trabajo es aplicar técnicas de Inteligencia Artificial y Big Data al sector de la cosmética para poder encontrar patrones y relaciones entre los cosméticos y los productos químicos que contienen, así como encontrar una clasificación de dichos productos químicos para poder tener un mayor control sobre de qué están compuestos los cosméticos que consumimos. \\




%------------------------------------------------    
\section{Objetivos}
\label{sec:goals}

Los objetivos principales de este trabajo son:
\begin{itemize}
 \item Encontrar una clasificación de los productos químicos presentes en los cosméticos.
 \item Encontrar qué productos químicos son los más frecuentes en los cosméticos, así como los cosméticos que mayor productos químicos presentan.
 \item Obtener una predicción de la cantidad de productos químicos dañinos que contendrán los futuros cosméticos.
\end{itemize}
























