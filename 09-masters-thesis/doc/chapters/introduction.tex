% Chapter Template

\chapter{Introducción y antecedentes} % Main chapter title
\label{chap:introduction} % Change X to a consecutive number; for referencing this chapter elsewhere, use \ref{ChapterX}

%------------------------------------------------
\section{Introducción}

La preocupación del ser humano por su aspecto y su cuidado es algo que se ha ido manteniendo a lo largo de los siglos, pues aunque hoy en día la industria cosmética parezca algo tecnológico y sofisticado, las primeras civilizaciones ya se preocupaban por ello. Precisamente, en el Antiguo Egipto se encuentran los primeros vestigios de la elaboración y utilización de diferentes productos cosméticos, utilizando para ello productos naturales, como plantas aromáticas. \\

La cosmética es uno de los sectores que mayor auge ha vivido durante las últimas décadas. Entre los siglos XVI al XVIII se produjo un gran desarrollo de los cosméticos y se introdujeron numerosos productos nuevos, aún fabricados principalmente a base de plantas. Pero fue a partir de principios del siglo XX cuando los cosméticos se popularizarían, hasta convertirse hoy en un producto casi imprescindible en la mayoría de los hogares. \\

Hoy en día, los cosméticos han vuelto a incorporar productos químicos dentro de sus fórmulas y se utilizan miles de estos compuestos a los que se le atribuyen multitud de propiedades. Sin embargo, varios cientificos y organizaciones han levantado la voz de alerta sobre el impacto de estos compuestos, pues en un alto porcentaje no han sido analizados para saber el daño de estos productos sobre las personas.



%------------------------------------------------
\section{Antecedentes}

Este trabajo se embarca dentro del contexto de la aplicación de técnicas Big Data e Inteligencia Artificial al sector de la cosmética. Existen varios productos y prototipos de productos que aplican al sector cosmético este tipo de técnicas, como por ejemplo el servicio Identité \citep{identite}, que utiliza técnicas de Inteligencia Artificial y Big Data para \textit{``Crear paquetes de belleza y productos para el cuidado de la piel muy personalizados.''}. 





%------------------------------------------------
\section{Estructura de la memoria}

Esta memoria se encuentra estructurada en seis capítulos, en los que se detalla lo siguiente:

\begin{itemize}
 \item \textbf{Capítulo 1}. El primer capítulo es el presente capítulo y en él se realiza la introducción al trabajo, los antecedentes al trabajo y la estructura que tiene la memoria.
 
 \item \textbf{Capítulo 2}. En el segundo capítulo se expone la hipótesis del trabajo y se detallan los objetivos del mismo.
 
 \item \textbf{Capítulo 3}. En el tercer capítulo se detalla el dataset con el que se va a realizar este trabajo, describiendo sus características y seleccionando aquellas más importantes, y los métodos y técnicas que se van a aplicar al dataset. 
 
 \item \textbf{Capítulo 4}. El cuarto capítulo es el capítulo central del trabajo, donde se expone la aplicación de las técnicas descritas en el capítulo 3 y los resultados obtenidos.
 
 \item \textbf{Capítulo 5}. En el quinto capítulo se analiza y se discute acerca de los resultados y los modelos obtenidos en el capítulo 4.
 
 \item \textbf{Capítulo 6}. Por último, el capítulo 6 detalla las conclusiones a las que ha llevado la realización de este trabajo, así como las conclusiones personales del autor.
\end{itemize}



















