% Chapter Template

\chapter{Material y métodos} % Main chapter title
\label{chap:material-methods} % Change X to a consecutive number; for referencing this chapter elsewhere, use \ref{ChapterX}


%------------------------------------------------    
\section{Introducción}

En esta sección se detallará, en primer lugar, el dataset utilizado para el desarrollo de este trabajo, la descripción de sus características y la selección de las características más importantes. En segundo lugar, se detallarán los métodos y técnicas que se han aplicado al dataset para conseguir los objetivos de este trabajo.





%------------------------------------------------    
\section{Material}
\label{sec:material}

Para la realización de este trabajo se ha hecho uso del dataset público \tabhead{Chemicals in Cosmetics} \citep{dataset} proporcionado por \tabhead{HealthData} \citep{healthdata}. \\

Para todos los productos cosméticos vendidos en California, la Ley de Cosméticos Seguros de California requiere que el fabricante, empacador y/o distribuidor nombrado en la etiqueta del producto proporcione una lista al Programa de Cosméticos Seguros de California \textit{(California Safe Cosmetics Program (CSCP))} perteneciente al Departamento de Salud Pública de California \textit{(California Department of Public Health (CDPH))} de todos los productos cosméticos que contengan cualquier ingrediente conocido o sospechoso de causar cáncer, defectos de nacimiento u otros daños al desarrollo o reproductivos. El CSCP mantiene una lista de ingredientes ``reportables''. Las compañías con ingredientes reportables en sus productos deben enviar información al Programa de Cosméticos Seguros de California si la compañía:

\begin{itemize}
 \item Tiene ventas anuales agregadas de productos cosméticos de un millón de dólares o más, y
 \item Ha vendido productos cosméticos en California a partir del 1 de enero de 2007.
\end{itemize}

El CSCP mantiene un sistema de informes \textit{online} para que las empresas informen sobre productos e ingredientes reportables, generando así el dataset utilizado para este trabajo. Los datos reflejan información que ha sido reportada al CSCP. No se incluyen todos los productos que contengan carcinógenos o tóxicos para el desarrollo o la reproducción, debido a que las compañías no los informan.



%------------------------------------------------    
\subsection{Descripción del dataset}
\label{sec:dataset-description}

El dataset proporcionado por el CSCP es un dataset en formato CSV con un histórico desde el año 2009 y que se va actualizando cada 3 o 4 días. En las fechas de realización de este trabajo (última fecha reportada: 21/02/2019), el dataset consta de:

\begin{itemize}
 \item 97.760 registros.
 \item 22 columnas (características).
\end{itemize}

En la Tabla \ref{tab:features} se describen cada una de las características. La agrupación de las características \code{CDPHId}, \code{CSFId}, \code{SubCategoryId} y \code{ChemicalId} forma la clave primaria de este dataset.


%------------------------------------------------    
\subsection{Selección de características importantes}

No todas las características descritas en la Tabla \ref{tab:features} son necesarias para aplicar las técnicas descritas en la sección \ref{sec:methods} ya que algunas de ellas son informativas o no son relevantes a la hora de aplicar dichas técnicas. \\

Se van a tener en cuenta todas las características de formato Fecha, exceptuando las características \code{DiscontinuedDate} y \code{ChemicalDateRemoved}, ya que solamente interesan aquellos productos que no hayan sido eliminados. Además, se van a utilizar las características \code{SubCategoryId} y \code{CASId}, pues son los identificadores de los cosméticos y los productos químicos, respectivamente. Se utiliza \code{SubCategoryId} en vez de \code{PrimaryCategoryId} debido a que la primera es más específica que la segunda y a partir de la primera se puede sacar la segunda (pues está contenido en ella). Y por supuesto, \code{ChemicalCount}, que indica la cantidad de productos químicos. \\

Así pues, la Tabla \ref{tab:features-selection} muestra las características que se van a tener en cuenta:

\begin{table}[!th]
\begin{tabular}{@{}l@{}}
\toprule
Nombre                         \\ \midrule
\code{SubCategoryId}           \\
\code{CASId}                   \\
\code{InitialDateReported}     \\
\code{MostRecentDateReported}  \\ 
\code{ChemicalCreatedAt}       \\
\code{ChemicalUpdatedAt}       \\ 
\code{ChemicalCount}           \\
\bottomrule
\end{tabular}
\centering
\caption{Selección de características importantes.}
\label{tab:features-selection}
\end{table}



\begin{table}[]
\begin{tabular}{@{}lll@{}}
\toprule
Nombre                 & Formato & Definición \\ \midrule
\code{CDPHId}          & Texto & Número identificativo interno del CDPH \\ & & para el producto. \\
\code{ProductName}     & Texto & Nombre del producto introducido por \\ & & el fabricante, empacador y/o distribuidor. \\
\code{CSFId}           & Texto & Número identificativo interno del CDPH \\ & & para el color/aroma/sabor. \\
\code{CSF}             & Texto & Color, aroma y/o sabor introducido por \\ & & el fabricante, empacador y/o distribuidor. \\                                                                                                                                             
\code{CompanyId}       & Texto & Número identificativo interno del CDPH \\ & & para la compañía. \\
\code{CompanyName}     & Texto & Nombre de la compañía introducido por \\ & & el fabricante, empacador y/o distribuidor. \\
\code{BrandName}       & Texto & Nombre de la marca introducido por \\ & & el fabricante, empacador y/o distribuidor. \\
\code{PrimaryCategoryId}  & Texto & Número identificativo interno del CDPH \\ & & para la categoría. \\
\code{PrimaryCategory}    & Texto & Tipo de producto (13 categorías primarias). \\
\code{SubCategoryId}      & Texto & Número identificativo interno del CDPH \\ & & para la subcategoría. \\
\code{SubCategory}        & Texto & Tipo de producto dentro de una de las \\ & & categorías primarias. \\
\code{CASId}              & Texto & Número identificativo interno del CDPH \\ & & para el producto químico. \\
\code{CasNumber}          & Texto & Número identificativo del producto \\ & & químico seleccionado por el fabricante, \\ & & empacador y/o distribuidor. \\ 
\code{ChemicalId}         & Texto & Número identificativo interno del CDPH \\ & & para el registro específico del producto \\ & & químico en ese cosmético. \\
\code{ChemicalName}       & Texto & Nombre del producto químico \\ & & seleccionado por el fabricante, empacador \\ & & y/o distribuidor. \\ 
\code{InitialDateReported}     & Fecha & Fecha en la que el cosmético fue \\ & & reportado por primera vez al CDPH. \\
\code{MostRecentDateReported}  & Fecha & Fecha de la última modificación del perfil \\ & & del cosmético por el fabricante, \\ & & empacador y/o distribuidor. \\ 
\code{DiscontinuedDate}        & Fecha & Si aplica, fecha en la que el cosmético fue \\ & & interrumpido. \\ 
\code{ChemicalCreatedAt}       & Fecha & Fecha en la que el producto químico fue \\ & & reportado al CDPH por primera vez en \\ & & ese cosmético. \\
\code{ChemicalUpdatedAt}       & Fecha & Fecha de la última modificación del perfil \\ & & del producto químico por el fabricante, \\ & & empacador y/o distribuidor. \\ 
\code{ChemicalDateRemoved}     & Fecha & Si aplica, fecha en la que el producto \\ & & químico fue eliminado del cosmético. \\ 
\code{ChemicalCount}           & Número & Número total de productos químicos \\ & &  reportados en ese cosmético \\
\bottomrule
\end{tabular}
\centering
\caption{Descripción de las características del dataset.}
\label{tab:features}
\end{table}


\newpage
%------------------------------------------------    
\section{Métodos}
\label{sec:methods}

Este trabajo se va a realizar utilizando el lenguaje de programación Python y como entorno de desarrollo Jupyter Notebook. Los métodos que se van a utilizar para la consecución de los objetivos descritos en la sección \ref{sec:goals} son:


%------------------------------------------------    
\subsection{Obtención del dataset}

Como se ha comentado en la sección \ref{sec:material}, el dataset se haya ubicado en la plataforma HealthData y además es incremental \citep{dataset}, por lo que para obtener dicho dataset se va a hacer uso de la API \citep{healthdata-api} que proporciona HealhData. \\



%------------------------------------------------    
\subsection{Clustering}

Para poder obtener la clasificación de los productos químicos presentes en los cosméticos, se van a aplicar técnicas de Clustering. Concretamente, se va a utilizar la implementación del algoritmo K-Means proporcionada por Scikit-learn \citep{scikit-learn}.



%------------------------------------------------    
\subsection{Data Analysis}

Para poder encontrar qué productos químicos son más frecuentes, así como aquellos cosméticos que presentan mayor número de productos químicos, se van a aplicar técnicas de Data Science como el Análisis Exploratorio de Datos \textit{(Exploratory Data Analysis (EDA))} \citep{eda}. La aplicación de EDA se va a realizar haciendo uso de los resultados obtenidos tras la clusterización.


%------------------------------------------------    
\subsection{Forecasting}

Para poder obtener la predicción de la cantidad de productos químicos que contendrán los futuros cosméticos, se van a aplicar técnicas de Forecasting. Concretamente, se va a utilizar la implementación del modelo ARIMA proporcionada por StatsModels \citep{arima}.




