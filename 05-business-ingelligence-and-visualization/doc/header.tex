\setcounter{secnumdepth}{3} % Para enumerar hasta subsubsection
\setcounter{tocdepth}{3} % Para enumerar hasta subsubsection 
 
\usepackage{indentfirst} % Paquete para iniciar el primer párrafo de una sección con sangría
\usepackage[utf8]{inputenc}
\usepackage[none]{hyphenat} %Paquete para indicar que no separe las palabras
\usepackage{graphicx} %Paquete para incluir imágenes
\usepackage{fancyhdr} %Paquete para modificar los pies de página y encabezados
\usepackage{listings} %Paquete para poner código de programación
\usepackage{color} %Paquete para definir los colores
\usepackage[hidelinks]{hyperref} %Paquete para que el índice tenga referencia
\usepackage{wrapfig} %Paquete para poner las imágenes con el texto en un lado.
\usepackage{multicol} %Paquete para poner texto en dos columnas
\usepackage{enumitem} %Paquete para poner en enumerate: 1.2, 1.3,...
\usepackage[nottoc]{tocbibind} %Paquete para incluir la bibliografía en el índice
\usepackage{colortbl} %Paquete para poner colores en las celdas
\usepackage{tocloft} %Paquete para personalizar el índice
\usepackage{textcomp} %Paquete para insertar símbolos raros
\usepackage{amssymb} %Paquete para insertar flechas raras
\usepackage{enumitem} %Paquete pare personalizar los enumerate
%\usepackage{kbordermatrix} %Parquete para poner bordes en las matrices

\definecolor{blue-violet}{rgb}{0.54,0.17,0.89}
\definecolor{darkcerulean}{rgb}{0.03, 0.27, 0.49}
\definecolor{ceruleanblue}{rgb}{0.16, 0.32, 0.75}
\definecolor{dkgreen}{rgb}{0,0.6,0}
\definecolor{gray}{rgb}{0.5,0.5,0.5}
\definecolor{mauve}{rgb}{0.58,0,0.82}
\definecolor{gray97}{gray}{.97}
\definecolor{gray75}{gray}{.75}
\definecolor{gray45}{gray}{.45}

\lstset{frame=Ltb,
  language=SQL,
  framerule=0pt,
  aboveskip=0.5cm,
  framextopmargin=3pt,
  framexbottommargin=3pt,
  framexleftmargin=0.4cm,
  framesep=0pt,
  rulesep=.4pt,
  backgroundcolor=\color{gray97},
  rulesepcolor=\color{black},
  %
  stringstyle=\color{mauve},
  showstringspaces = false,
  basicstyle=\scriptsize\ttfamily,
  commentstyle=\color{gray45},
  keywordstyle=\color{darkcerulean},
  %
  numbers=left,
  numbersep=15pt,
  numberstyle=\tiny,
  numberfirstline = false,
  breaklines=true,
  escapeinside=||,
  %
  literate={«}{{\guillemotleft}}1
           {»}{{\guillemotright}}1
           {é}{{\'e}}1
           {í}{{\'i}}1
           {ó}{{\'o}}1
           {ú}{{\'u}}1
           {á}{{\'a}}1
           {ñ}{{\~n}}1
           {Ñ}{{\~N}}1
           {¿}{{?`}}1
}

% minimizar fragmentado de listados
\lstnewenvironment{listing}[1][]
{\lstset{#1}\pagebreak[0]}{\pagebreak[0]}

\renewcommand{\contentsname}{Índice}
\renewcommand{\partname}{TEMA}
\renewcommand{\chaptername}{Tarea}
\renewcommand{\thesection}{\arabic{section}}
\renewcommand{\listtablename}{Índice de tablas}
\renewcommand{\tablename}{Tabla}
\renewcommand{\figurename}{Figura}
\renewcommand{\bibname}{Bibliografía}
% \renewcommand{\cftsecfont}{\bfseries} % Poner las secciones en negrita en el índice

% Personalización del itemize
\renewcommand{\labelitemi}{\textendash}
\renewcommand{\labelitemii}{\textperiodcentered}
\renewcommand{\labelitemiii}{\textasciicircum}
